% Created 2015-11-18 Wed 21:02
\documentclass[11pt]{article}
\usepackage[utf8]{inputenc}
\usepackage{lmodern}
\usepackage[T1]{fontenc}
\usepackage{fixltx2e}
\usepackage{graphicx}
\usepackage{longtable}
\usepackage{float}
\usepackage{wrapfig}
\usepackage{rotating}
\usepackage[normalem]{ulem}
\usepackage{amsmath}
\usepackage{textcomp}
\usepackage{marvosym}
\usepackage{wasysym}
\usepackage{amssymb}
\usepackage{amsmath}
\usepackage[version=3]{mhchem}
\usepackage[numbers,super,sort&compress]{natbib}
\usepackage{natmove}
\usepackage{url}
\usepackage{minted}
\usepackage{underscore}
\usepackage[linktocpage,pdfstartview=FitH,colorlinks,
linkcolor=blue,anchorcolor=blue,
citecolor=blue,filecolor=blue,menucolor=blue,urlcolor=blue]{hyperref}
\usepackage{attachfile}
\usepackage{todonotes}
\usepackage{graphicx}
\usepackage{subfigure}
\date{\today}
\title{org-element-explorer}
\begin{document}

\section{Functional and display math}
\label{sec-1}
I have been thinking about a way to have functional and readable mathematics in technical documents. It has always bothered me that I have to write a \LaTeX{} version of an equation, and then a separate implementation of the equation in code somewhere. At least twice these separate representations have not agreed!

One solution might be if my functional code could be converted to \LaTeX{} easily. I explore one simple approach to this here. It is somewhat inspired by this work here \url{http://oremacs.com/2015/01/23/eltex/} on writing \LaTeX{} in emacs-lisp, and from my work with org-mode in mixing narrative text, \LaTeX{} and code.

The idea is to use emacs-lisp for the code, so it is functional, but provide an alternative output for the \emph{same code} for a document conversion. In other words, we accept there is more than one version we need: a functional version for working, and a consumption version for presentation. We will generate the consumption version from the functional version.

I know emacs-lisp is not ideal for mathematics the way we are accustomed to seeing it, but it enables the idea I want to explore here so we will try it.

Here is the simplest example I could come up with for functional math. We can run it ourselves, and verify it is correct.

\begin{minted}[frame=lines,fontsize=\scriptsize,linenos]{common-lisp}
(+ 1 2 3)
\end{minted}

\begin{verbatim}
6
\end{verbatim}

Now, I can change the meaning of this code temporarily, so that it not only evaluates the form, but also represents the equation and result in \LaTeX{} code. If this was incorporated into a preprocessor of the document, we could have a functional version representing our equations, in code form, and a presentation version generated from this version.

\begin{minted}[frame=lines,fontsize=\scriptsize,linenos]{common-lisp}
(cl-flet ((+ (lambda (&rest args)
               (concat
                "$"
                (mapconcat #'number-to-string args " + ")
                " = "
                (number-to-string (eval `(+ ,@args)))
                "$"))))
  (+ 1 2 3))
\end{minted}

$1 + 2 + 3 = 6$

Getting this to a truly functional approach would require a lot of work, basically creating transformation functions for many, many kinds of mathematical functions, and a lot of other kinds of logic. For example, (+ 1 2 (+ 3 4)) would not render correctly with the code above.

Here is an example that generates a fraction from a division.
\begin{minted}[frame=lines,fontsize=\scriptsize,linenos]{common-lisp}
(cl-flet ((/ (lambda (&rest args)
               (format
                "$\\frac{%s}{%s} = %s$"
                (car args)
                (mapconcat 'number-to-string (cdr args) " \\cdot ")
                (number-to-string (eval `(/ ,@args)))))))
  (/ 1.0 2.0 3.0))
\end{minted}

$\frac{1.0}{2.0 \cdot 3.0} = 0.16666666666666666$
% Emacs 25.0.50.1 (Org mode 8.2.10)
\end{document}